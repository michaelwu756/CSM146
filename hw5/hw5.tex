\documentclass[12pt]{article}
\usepackage{amsmath}
\begin{document}
\title{Computer Science M146, Homework 5}
\date{March 15th, 2018}
\author{Michael Wu\\UID: 404751542}
\maketitle

\section*{Problem 1}

\paragraph{a)}

We lose the ordering of the documents, as a document \(D_1=\{a,b,c\}\) is treated as equivalent to the document \(D_2=\{a,c,b\}\). We only
care about the number of words in the document, not the order they appear in.

\paragraph{b)}

\begin{align*}
        \log \operatorname{P}(D_i, y_i) &= \log\left(\left(\operatorname{P}(D_i|y_i=1)\operatorname{P}(y_i=1)\right)^{y_i}
                \left(\operatorname{P}(D_i|y_i=0)\operatorname{P}(y_i=0)\right)^{1-y_i}\right)\\
        &=\log \left(\operatorname{P}(D_i|y_i=1)^{y_i}\theta^{y_i}\operatorname{P}(D_i|y_i=0)^{1-y_i}(1-\theta)^{1-y_i}\right)\\
        &=y_i(\log \theta + \log \operatorname{P}(D_i|y_i=1))\\
        &\qquad+(1-y_i) (\log (1-\theta) + \log \operatorname{P}(D_i|y_i=0))\\
        &=y_i\left(\log \theta + \log \frac{n!}{a_i!b_i!c_i!}\alpha_1^{a_i}\beta_1^{b_i}\gamma_1^{c_i}\right)\\
        &\qquad+(1-y_i) \left(\log(1-\theta)+ \log \frac{n!}{a_i!b_i!c_i!}\alpha_0^{a_i}\beta_0^{b_i}\gamma_0^{c_i}\right)\\
        &=y_i\left(\log \theta + \log \frac{n!}{a_i!b_i!c_i!}+a_i\log\alpha_1+b_i\log\beta_1+c_i\log\gamma_1\right)\\
        &\qquad+(1-y_i) \left(\log(1-\theta)+ \log \frac{n!}{a_i!b_i!c_i!}+a_i\log\alpha_0+b_i\log\beta_0\right.\\
        &\qquad\left.\phantom{\frac{0}{0}}+c_i\log\gamma_0\right)
\end{align*}

\begin{align*}
        \log \operatorname{P}(D_i, y_i) &= \log \frac{n!}{a_i!b_i!c_i!} + y_i(\log \theta +a_i\log\alpha_1+b_i\log\beta_1+c_i\log\gamma_1)\\
        &\qquad+(1-y_i)(\log(1-\theta)+a_i\log\alpha_0+b_i\log\beta_0+c_i\log\gamma_0)
\end{align*}

\paragraph{c)}

We can find the maximum likelihood estimate for each of our parameters by finding
\[\max_{\alpha_1,\beta_1,\gamma_1,\alpha_0,\beta_0,\gamma_0}\sum_{i=1}^m\log \operatorname{P}(D_i, y_i)\]
Our maximum likelihood estimate occurs when
\[\frac{\partial}{\partial \alpha_1}\sum_{i=1}^m\log \operatorname{P}(D_i, y_i)=0\]
Because we are given that \(\alpha_1+\beta_1+\gamma_1=1\), we know that \(\beta_1\) and \(\gamma_1\) are functions of \(\alpha_1\). So we can let
\(\beta_1=1-\alpha_1-\gamma_1\). Taking the partial derivative yields
\begin{align*}
        \frac{\partial}{\partial \alpha_1}\sum_{i=1}^m\log \operatorname{P}(D_i, y_i)&=\frac{\partial}{\partial \alpha_1}\sum_{i=1}^m y_i(a_i\log\alpha_1+b_i\log\beta_1)\\
        &=\frac{\partial}{\partial \alpha_1}\sum_{i=1}^m y_i(a_i\log\alpha_1+b_i\log(1-\alpha_1-\gamma_1))\\
        &=\sum_{i=1}^my_i\left(\frac{a_i}{\alpha_1}-\frac{b_i}{1-\alpha_1-\gamma_1}\right)\\
        &=\sum_{i=1}^my_i(a_i(1-\alpha_1-\gamma_1)-b_i\alpha_1)\\
        &=(1-\gamma_1)\sum_{i=1}^m y_ia_i-\alpha_1\sum_{i=1}^m y_i(a_i+b_i)\\
\end{align*}
Then we can solve for zero which gives
\[\alpha_1=\frac{\sum_{i=1}^m y_ia_i}{\sum_{i=1}^m y_i(a_i+b_i)}(1-\gamma_1)=C_1(1-\gamma_1)\]
Using the same process with the partial derivative for \(\gamma_1\) yields
\begin{align*}
        \frac{\partial}{\partial \gamma_1}\sum_{i=1}^m\log \operatorname{P}(D_i, y_i)&=\frac{\partial}{\partial \gamma_1}\sum_{i=1}^m y_i(c_i\log\gamma_1+b_i\log\beta_1)\\
        &=\sum_{i=1}^my_i\left(\frac{c_i}{\gamma_1}-\frac{b_i}{1-\alpha_1-\gamma_1}\right)\\
        &=\sum_{i=1}^my_i(c_i(1-\alpha_1-\gamma_1)-b_i\gamma_1)\\
        &=(1-\alpha_1)\sum_{i=1}^m y_ic_i-\gamma_1\sum_{i=1}^m y_i(b_i+c_i)\\
\end{align*}
Setting this equal to zero gives us
\[\gamma_1=\frac{\sum_{i=1}^m y_ic_i}{\sum_{i=1}^m y_i(b_i+c_i)}(1-\alpha_1)=C_2(1-\alpha_1)\]
and thus
\begin{align*}
        \alpha_1&=\frac{C_1(1-C_2)}{1-C_1C_2}\\
        \beta_1&=\frac{(1-C_1)(1-C_2)}{1-C_1C_2}\\
        \gamma_1&=\frac{C_2(1-C_1)}{1-C_1C_2}
\end{align*}
where
\begin{align*}
        C_1&=\frac{\sum_{i=1}^m y_ia_i}{\sum_{i=1}^m y_i(a_i+b_i)}\\
        C_2&=\frac{\sum_{i=1}^m y_ic_i}{\sum_{i=1}^m y_i(b_i+c_i)}
\end{align*}
Similarly for \(\alpha_0\), \(\beta_0\), and \(\gamma_0\), taking partial derivatives and setting them to zero yields
\begin{align*}
        \alpha_0&=\frac{C_3(1-C_4)}{1-C_3C_4}\\
        \beta_0&=\frac{(1-C_3)(1-C_4)}{1-C_3C_4}\\
        \gamma_0&=\frac{C_4(1-C_3)}{1-C_3C_4}
\end{align*}
where
\begin{align*}
        C_1&=\frac{\sum_{i=1}^m (1-y_i)a_i}{\sum_{i=1}^m (1-y_i)(a_i+b_i)}\\
        C_2&=\frac{\sum_{i=1}^m (1-y_i)c_i}{\sum_{i=1}^m (1-y_i)(b_i+c_i)}
\end{align*}
because of the symmetry of our log likelihood function \(\log \operatorname{P}(D_i, y_i)\).

\section*{Problem 2}

\paragraph{a)}

\paragraph{b)}

\paragraph{c)}

\section*{Problem 3}

\paragraph{a)}

\paragraph{b)}

\paragraph{c)}

\paragraph{d)}

\paragraph{e)}

\paragraph{f)}

\end{document}