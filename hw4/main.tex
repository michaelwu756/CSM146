\documentclass[12pt]{article}
\usepackage{amsmath}
\begin{document}
\title{Computer Science M146, Homework 4}
\date{March 8th, 2018}
\author{Michael Wu\\UID: 404751542}
\maketitle

\section*{Problem 1}

We have the following data
\begin{center}
        \begin{tabular}{|c|c|c|c|}
                \hline
                {\em i}  & $x$  & $y$ & Label \\
                \hline
                {\em 1}  & 0  & 8 & $-$ \\
                \hline
                {\em 2}  & 1  & 4 & $-$ \\
                \hline
                {\em 3}  & 3  & 7 & $+$ \\
                \hline
                {\em 4}  & -2  & 1 & $-$ \\
                \hline
                {\em 5}  & -1  & 13 & $-$ \\
                \hline
                {\em 6}  & 9  & 11 & $-$ \\
                \hline
                {\em 7}  & 12 & 7 & $+$ \\
                \hline
                {\em 8}  & -7  & -1 & $-$ \\
                \hline
                {\em 9}  & -3  & 12 & $+$ \\
                \hline
                {\em 10} & 5  & 9 & $+$ \\
                \hline
        \end{tabular}
\end{center}
and after running our AdaBoost algorithm we get the following table.
\begin{center}
        \hspace*{-2cm}
        \begin{tabular}{|c|c||c|c|c|c||c|c|c|c|}
                \hline
                & & \multicolumn{4}{c||}{Hypothesis 1 (1st iteration)}
                & \multicolumn{4}{c|}{Hypothesis 2 (2nd iteration)} \\
                \cline{3-10}
                {\em i} & Label & $D_0$ & $f_1 \equiv $ & $f_2 \equiv $ & $h_1\equiv$ & $D_1$ &  $f_1 \equiv $ & $f_2 \equiv $ & $h_2 \equiv $ \\
                & & & [$x > 2$] & [$y > 6$] & [$x > 2$] & & [$x > 11$] & [$y > 11 $] & [$y > 11$] \\
                \tiny{(1)} & \tiny{(2)} & \tiny{(3)} & \tiny{(4)} &  \tiny{(5)} & \tiny{(6)} & \tiny{(7)} & \tiny{(8)} & \tiny{(9)} & \tiny{(10)}\\
                \hline \hline
                {\em 1} & $-$ & \(\frac{1}{10}\) & \(-\) & \(+\) & \(-\) & \(\frac{1}{16}\) & \(-\) & \(-\) & \(-\) \\
                \hline
                {\em 2} & $-$ & \(\frac{1}{10}\) & \(-\) & \(-\) & \(-\) & \(\frac{1}{16}\) & \(-\) & \(-\) & \(-\) \\
                \hline
                {\em 3} & $+$ & \(\frac{1}{10}\) & \(+\) & \(+\) & \(+\) & \(\frac{1}{16}\) & \(-\) & \(-\) & \(-\) \\
                \hline
                {\em 4} & $-$ & \(\frac{1}{10}\) & \(-\) & \(-\) & \(-\) & \(\frac{1}{16}\) & \(-\) & \(-\) & \(-\) \\
                \hline
                {\em 5} & $-$ & \(\frac{1}{10}\) & \(-\) & \(+\) & \(-\) & \(\frac{1}{16}\) & \(-\) & \(+\) & \(+\) \\
                \hline
                {\em 6} & $-$ & \(\frac{1}{10}\) & \(+\) & \(+\) & \(+\) & \(\frac{1}{4}\) & \(-\) & \(-\) & \(-\) \\
                \hline
                {\em 7} & $+$ & \(\frac{1}{10}\) & \(+\) & \(+\) & \(+\) & \(\frac{1}{16}\) & \(+\) & \(-\) & \(-\) \\
                \hline
                {\em 8} & $-$ & \(\frac{1}{10}\) & \(-\) & \(-\) & \(-\) & \(\frac{1}{16}\) & \(-\) & \(-\) & \(-\) \\
                \hline
                {\em 9} & $+$ & \(\frac{1}{10}\) & \(-\) & \(+\) & \(-\) & \(\frac{1}{4}\) & \(-\) & \(+\) & \(+\) \\
                \hline
                {\em 10} & $+$ & \(\frac{1}{10}\) & \(+\) & \(+\) & \(+\) & \(\frac{1}{16}\) & \(-\) & \(-\) & \(-\) \\
                \hline
        \end{tabular}
        \hspace*{-2cm}
\end{center}
For the first iteration we get the error and vote
\[\epsilon_1 = \frac{2}{10}\]
\[\alpha_1 = \frac{1}{2}\log_2\left(\frac{1-\frac{2}{10}}{\frac{2}{10}}\right) = 1\]
which allows us to generate the next weights
\[D_1(i)=\frac{D_0(i)}{Z_1}2^{-\alpha_1y_ih_1(x_i)}=2^{-3-y_ih_1(x_i)}=\begin{cases}\frac{1}{4}&\text{if correct}\\ \frac{1}{16}&\text{if incorrect}\end{cases}\]
For our second iteration we get the error and vote
\[\epsilon_2 = \frac{1}{4}\]
\[\alpha_2 = \frac{1}{2}\log_2\left(\frac{1-\frac{1}{4}}{\frac{1}{4}}\right) = \frac{\log_2(3)}{2}\]
and combining these two gives us our final hypothesis
\[H_\text{final}(x,y)=\operatorname{sgn}\left(h_1(x,y)+\frac{\log_2(3)}{2}h_2(x,y)\right)=\begin{cases}1&\text{if }x>2\\ -1&\text{if }x\leq 2\end{cases}\]
which is the same as hypothesis \(h_1\) because \(\alpha_1>\alpha_2\), so \(h_1\) effectively overpowers \(h_2\).

\section{Multi-class classification - 60 points}

Consider a multi-class classification problem with $k$ class
labels $\{1, 2, \ldots k\}$. Assume that we are given $m$
examples, labeled with one of the $k$ class labels. Assume, for
simplicity, that we have $m/k$ examples of each type.

Assume that you have a learning algorithm $L$ that can be used
to learn Boolean functions. (E.g., think about $L$ as the
Perceptron algorithm). We would like to explore several ways to
develop learning algorithms for the multi-class classification
problem.

There are two schemes to use the algorithm $L$ on the given data set, and produce a multi-class classification:
\begin{itemize}
\item {\bf One vs.~All:} For every label $i \in [1,k]$, a classifier is learned over the following data set: the examples labeled with the label $i$ are considered ``positive'', and examples labeled with any other class $j \in [1,k], j \neq i$ are considered ``negative''.
\item {\bf All vs.~All:} For every pair of labels $\langle i, j \rangle$, a classifier is learned over the following data set: the examples labeled with one class $i \in [1,k]$ are considered ``positive'', and those labeled with the other class $j \in [1,k], j \neq i$ are considered ``negative''.
\end{itemize}
%
\vspace{-3mm}
\begin{enumerate}
\item {\bf [20 points]} For each of these two schemes, answer the following:
\begin{enumerate}
\item How many classifiers do you learn?
\item How many examples do you use to learn each classifier within the scheme?
\item How will you decide the final class label (from \{1, 2, \ldots, k\}) for each example?
\item What is the computational complexity of the training process?
\end{enumerate}
\item {\bf [5 points]} Based on your analysis above of two schemes individually, which scheme would you prefer? Justify.
\item {\bf [5 points]} You could also use a \textsc{KernelPerceptron} for a two-class classification. We could also use the algorithm to learn a multi-class classification. Does using a \textsc{KernelPerceptron} change your analysis above? Specifically, what is the computational complexity of using a \textsc{KernelPerceptron} and which scheme would you prefer when using a \textsc{KernelPerceptron}? 

\item {\bf [10 points]} We are given a magical black-box binary classification algorithm (we don’t know how it works, but it just does!) which has a learning time complexity of O($dn^2$), where $n$ is the total number of training examples supplied (positive+negative) and $d$ is the dimensionality of each example.
What are the overall training time complexities of the all-vs-all and the one-vs-all
paradigms, respectively, and which training paradigm is most efficient?

\item {\bf [10 points]} We are now given another magical black-box binary classification algorithm (wow!) which has a learning time complexity of O($d^2 n$), where $n$ is the total number of training examples supplied (positive+negative) and $d$ is the dimensionality of each example.
What are the overall training time complexities of the all-vs-all and the one-vs-all paradigms, respectively, and which training paradigm is most efficient, when using this new classifier?

\item {\bf [10 points]} Suppose we have learnt an all-vs-all multi-class classifier and now want to proceed to predicting labels on unseen examples.

We have learnt a simple linear classifier with a weight vector of dimensionality $d$ for each of the $ m(m-1)/2$ classes ($w_i^T x = 0$ is the simple linear classifier hyperplane for each  $i =[1, \cdots , m(m-1)/2] )$

We have two evaluation strategies to choose from. For each example, we can:
\begin{itemize}
  \item \textbf{Counting}: Do all predictions then do a majority vote to decide class label
  \item \textbf{Knockout}: Compare two classes at a time, if one loses, never consider it
again. Repeat till only one class remains.
\end{itemize}
What are the overall evaluation time complexities per example for Counting and
Knockout, respectively?
\end{enumerate}


\end{document}