\documentclass[12pt]{article}
\usepackage{amsmath}
\usepackage{amsfonts}
\usepackage{enumerate}
\usepackage{pgfplots}
\usepackage{calc}
\usepackage{graphicx}
\usepackage{float}
\usepackage{hyperref}
\usepackage{chngpage}
\pgfplotsset{compat=1.12}
\hypersetup{colorlinks,urlcolor=blue}
\graphicspath{{./code/src/}}
\begin{document}
\title{Computer Science M146, Homework 2}
\date{February 6th, 2018}
\author{Michael Wu\\UID: 404751542}
\maketitle

\section*{Problem 1}

\paragraph{a)}

We have the following points
\[
        \begin{array}{c c c}
                x_1 & x_2 & y\\
                \hline
                -1 & -1 & -1\\
                -1 & 1 & -1\\
                1 & -1 & -1\\
                1 & 1 & 1
        \end{array}
\]
Initializing \(\pmb{\theta}=\left<0,0\right>\) and \(b=0\) and running the perceptron algorithm on this list from top to bottom yields
\(\pmb{\theta}=\left<1,1\right>\) and \(b=-1\). Then
\[y=\begin{cases}
        1 & \text{if } \theta_1x_1+\theta_2x_2+b\geq0\\
        -1 & \text{if } \theta_1x_1+\theta_2x_2+b<0
\end{cases}\]
which separates our training data perfectly. This is not the only unique way to separate our data, as if we chose \(\pmb{\theta}=\left<2,1\right>\)
and \(b=-2\) this would be a valid perceptron as well.

\paragraph{b)}

We have the following points
\[
        \begin{array}{c c c}
                x_1 & x_2 & y\\
                \hline
                -1 & -1 & -1\\
                -1 & 1 & 1\\
                1 & -1 & 1\\
                1 & 1 & -1
        \end{array}
\]
No valid perceptron exists, because this data is not linearly separable.

\section*{Problem 2}

\paragraph{a)}

\begin{align*}
        \frac{\partial J(\pmb{\theta})}{\partial \theta_j}
        &=-\frac{\partial}{\partial \theta_j}\sum_{n=1}^N \left[y_n \ln\left(\frac{1}{1+e^{-\pmb{\theta}^T x_n}}\right)
        +(1-y_n)\ln\left(\frac{e^{-\pmb{\theta}^T x_n}}{1+e^{-\pmb{\theta}^T x_n}}\right)\right]\\
        &=-\frac{\partial}{\partial \theta_j}\sum_{n=1}^N \left[-y_n \ln\left(1+e^{-\pmb{\theta}^T x_n}\right)
        +(1-y_n)\left(-\pmb{\theta}^T x_n-\ln\left(1+e^{-\pmb{\theta}^T x_n}\right)\right)\right]\\
        &=-\sum_{n=1}^N \left[y_nx_{nj}\frac{e^{-\pmb{\theta}^T x_n }}{1+e^{-\pmb{\theta}^T x_n}}
        +(1-y_n)\left(-x_{nj}+x_{nj}\frac{e^{-\pmb{\theta}^T x_n }}{1+e^{-\pmb{\theta}^T x_n}}\right)\right]\\
        &=-\sum_{n=1}^N x_{nj}\left[y_n\frac{e^{-\pmb{\theta}^T x_n }}{1+e^{-\pmb{\theta}^T x_n}}
        +(1-y_n)\left(\frac{e^{-\pmb{\theta}^T x_n }}{1+e^{-\pmb{\theta}^T x_n}}-1\right)\right]\\
        &=-\sum_{n=1}^N x_{nj}\left[\frac{e^{-\pmb{\theta}^T x_n }}{1+e^{-\pmb{\theta}^T x_n}}+y_n-1\right]\\
        &=-\sum_{n=1}^N x_{nj}\left(y_n-\frac{1}{1+e^{-\pmb{\theta}^T x_n}}\right)\\
        &=\sum_{n=1}^N x_{nj}\left(h_{\pmb{\theta}}(x_n)-y_n\right)
\end{align*}

\section*{Problem 3}

\paragraph{a)}

\[\nabla J(\theta_0,\theta_1)=2\sum_{n=1}^N \left[w_n(\theta_0+\theta_1x_{n,1}-y_n)\left<1, x_{n,1}\right>\right]\]

\paragraph{b)}

First we will write out the partial derivative with respect to \(\theta_0\) and set it to zero.

\begin{align*}
        \frac{\partial J}{\partial \theta_0}=2\sum_{n=1}^N \left[w_n(\theta_0+\theta_1x_{n,1}-y_n)\right]&=0\\
        \sum_{n=1}^N \left[w_n(\theta_0+\theta_1x_{n,1}-y_n)\right]&=0\\
        \theta_0\sum_{n=1}^N w_n + \theta_1\sum_{n=1}^N w_nx_{n,1} - \sum_{n=1}^N w_ny_n&=0\\
        a\theta_0 + b\theta_1 &= c
\end{align*}
where \(a=\sum_{n=1}^N w_n\), \(b=\sum_{n=1}^N w_nx_{n,1}\), and \(c=\sum_{n=1}^N w_ny_n\). Next we
will write out the partial derivative with respect to \(\theta_1\) and set it to zero.
\begin{align*}
        \frac{\partial J}{\partial \theta_1}=2\sum_{n=1}^N \left[x_{n,1}w_n(\theta_0+\theta_1x_{n,1}-y_n)\right]&=0\\
        \sum_{n=1}^N \left[x_{n,1}w_n(\theta_0+\theta_1x_{n,1}-y_n)\right]&=0\\
        \theta_0\sum_{n=1}^N x_{n,1}w_n + \theta_1\sum_{n=1}^N w_nx_{n,1}^2 - \sum_{n=1}^N x_{n,1}w_ny_n&=0\\
        b\theta_0 + d\theta_1 &= e
\end{align*}
where \(d=\sum_{n=1}^N w_nx_{n,1}^2\) and \(e=\sum_{n=1}^N x_{n,1}w_ny_n\). Then solving this system of equations gives
\begin{align*}
        \theta_0&=\frac{e-d\theta_1}{b}\\
        \frac{a}{b}(e-d\theta_1)+b\theta_1&=c\\
        \frac{ae}{b}+\left(b-\frac{ad}{b}\right)\theta_1&=c\\
        \frac{b^2-ad}{b}\theta_1&=c-\frac{ae}{b}\\
        \theta_1&=\frac{bc-ae}{b^2-ad}
\end{align*}
\begin{align*}
        \theta_0&=\frac{e-d\frac{bc-ae}{b^2-ad}}{b}\\
        \theta_0&=\frac{eb^2-ade-bcd+ade}{b^3-abd}\\
        \theta_0&=\frac{eb-cd}{b^2-ad}
\end{align*}
Then we have a minimum at
\[(\theta_0,\theta_1)=\frac{1}{b^2-ad}\left<eb-cd, bc-ae\right>\]

\section*{Problem 4}

\paragraph{a)}

Assume that our data set is finite, so we have for all \((x_i,y_i)\in D\) there exists a \(w\) and \(\theta\) such that
\[y_i(w^Tx_i+\theta)>l\]
where \(l\) is the margin between our separating line and the closest point in our data set. Then we can multiply our
weight vector \(w\) and our threshold value \(\theta\) by \(\frac{1}{l}\) to get
\[y_i\left(\frac{1}{l}w^Tx_i+\frac{\theta}{l}\right)>1\]
Thus we can find a weight vector and threshold value that creates an optimal solution to the linear program with \(\delta=0\).

\paragraph{b)}

If we have an optimal solution to the linear problem, we have
\[y_i\left(w^Tx_i+\theta\right)>1\]
for all \((x_i,y_i)\in D\). In order for this to happen, \(y_i\) must always have the same sign as \(w^Tx_i+\theta\). Thus
\[y_i=\begin{cases}
        1 & \text{if }w^Tx_i+\theta\geq 0\\
        -1 & \text{if }w^Tx_i+\theta<0
\end{cases}\]
for all \((x_i,y_i)\in D\), and so \(D\) is linearly separable.

\paragraph{c)}

If \(0<\delta<1\), then our data set is linearly separable, using the same argument as above. Otherwise we cannot say anything about
if our data set is linearly separable.

\paragraph{d)}

An optimal solution is \(w=0\) and \(\theta=0\). Then
\[y_i(w^Tx_i+\theta)\geq 0\]
for any data set. The problem with this is that our resulting linear separator tells us nothing about our data set, as everything is
zero.

\paragraph{e)}

A possible optimal solution is \(w=\left<1,1,1\right>\) and \(\theta=0\).

\end{document}